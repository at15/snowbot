\documentclass[11pt,letterpaper,twocolumn,oneside]{article}
\usepackage[letterpaper]{geometry}
\geometry{left=1.8cm, top=1.0cm, right=1.8cm, bottom=2.0cm, footskip=.5cm}
% FIXME: the font is really strange
% \usepackage[T1]{fontenc}
% https://tex.stackexchange.com/questions/39227/no-indent-in-the-first-paragraph-in-a-section
\usepackage{indentfirst}
% \setlength\parindent{0pt}

\usepackage{enumitem}

\usepackage{amsmath}
\usepackage{amsfonts}
\usepackage{amsthm}
\usepackage{mathtools}
\DeclarePairedDelimiter{\ceil}{\lceil}{\rceil}
\DeclarePairedDelimiter{\floor}{\lfloor}{\rfloor}
\usepackage{amssymb}

\usepackage{algorithm}
\usepackage{algorithmicx}
\usepackage{algpseudocode}

% http://tex.stackexchange.com/a/230427/114385
\usepackage{hyperref}

\usepackage{graphicx}
\graphicspath{ {images/} }

\usepackage[backend=biber]{biblatex}
\addbibresource{proposal.bib}

\usepackage{sectsty}
\sectionfont{\fontsize{12}{15}\selectfont}

\begin{document}

% from http://latex.org/forum/viewtopic.php?f=5&t=3032
\twocolumn[%
%  Title and authors
  \begin{center}
    {\large An empirical study of machine learning frameworks for building chatbot in cloud} \\
     \vspace{2ex}
     CMPS242 course project proposal Pinglei Guo et al.
  \end{center}
]

\section{Abstract}

Machine learning frameworks is now a new competing spot for tech giants,
especially cloud service providers with GPU or TPU\footnote{https://cloud.google.com/tpu/} support.
Main stream frameworks are all backed by companies,
Google created tensorflow, Amazon chose MXNet, Facebook created PyTorch and Microsoft built CNTK.
Though they are working on open format neural network format ONNX\footnote{http://onnx.ai/},
the cost of implementing and running same model using different framework on different cloud still varies.

We chose to build a chatbot using seq2seq in the experiment,
though it's hard to evaluate performance of a dialogue system\cite{liu2016not}.
It's close related to previous course projects and our main goal is to explore different frameworks
instead of reach good score in a well defined problem.
Also we plan to use it as a base for building a personal knowledge manage assistant in the future.

\section{Design}

This design

\printbibliography

\end{document}
